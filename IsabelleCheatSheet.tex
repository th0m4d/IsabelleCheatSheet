\documentclass{article}
\usepackage{amsmath}
\title{Semantics Cheat Sheet}
\author{Thomas Pieronczyk}

\begin{document}
\maketitle
\section{Semantics Summary}
\subsection{Proof Methods}
\begin{itemize}
	\item [\textbf{induction}] performs structural induction on some variable (if the type of the variable is a datatype).
	\item [\textbf{auto}] solves as many subgoals as it can, mainly by simplification (symbolic evaluation). "=" is used only from left to right.
	\item [\textbf{sorry}] Completes any proof. Allows Top-Down Development (Assume lemma first and proof it later).
	\item [\textbf{defer}] Takes the first subgoal and puts it to the end of the other subgoals.
	\item [\textbf{value}] Yields results of given function or term.
\end{itemize}

\subsection{Simplifications}

\begin{itemize}
	\item \textbf{apply(auto split: ...)}
	\item \textbf{apply(auto split: mydatatype.split)} If you want to prove a predicate for a certain datatype \texttt{mydatatype}, this will create all the single instances for this datatype that have to be proven to prove the overall thing.
	\item \textbf{apply(auto intro: ...)}
	\item \textbf{apply(auto split: ...)}
	\item \textbf{apply(auto split: split\_if\_asm)} Splits the if assumptions in the subgoals.
	\item \textbf{apply(rule ...)} Applies a rule, which is defined by an inductive predicate. That is, it matches the goal with something and replaces it by some (hopefully easier) premises.
	\item \textbf{apply(rule ... [where "\textless\textgreater"])} Applies an inductive rule and a substitution of the state. The universal quantifier is replaced with the specific state. In this case all variables are set to zero.
	\item \textbf{apply assumption} Scans the assumptions for something that looks like the conclusion. Useful in a situation
\end{itemize}

\subsection{General scheme}
\begin{verbatim}
lemma name: "..."
apply(...)
apply(...)
apply(...)

...

done

\end{verbatim}

\subsection{Use lemma as simplification rule}

\texttt{lemma mylemma[simp]: "..."} adds \texttt{mylemma} to the standard rules used for simplification. Simplification done ``from left to right''.

\texttt{apply (auto simp add: thelemma)} let's you use \texttt{thelemma} as simplification rule.

\subsection {Hints for isabella and proof general}
\begin{itemize}
	\item C-c C-f finds all theorems (or find\_theorems "\_ \~ \_").
        \item C-c C-n makes next proof step
        \item C-c C-u goes back in proof history (or at least tries to do)
        \item C-c C-p shows the current proof state
\end{itemize}

\end{document}
